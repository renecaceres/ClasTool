\documentclass[11pt]{beamer}
\usepackage[spanish]{babel}
\usepackage[utf8]{inputenc}
\usepackage{graphics}
\usepackage{listings}
\usepackage{array}
\usepackage{times}


\usetheme{CambridgeUS}
\usecolortheme{seagull}
\usefonttheme[onlylarge]{structurebold}

\lstset{%
     showstringspaces = false,
     basicstyle=\small\ttfamily,
   }

\author{Sebasti\'an Mancilla}
\institute[UTFSM]{Universidad T\écnica Federico Santa Mar\ía}
\date{\today}
\title{ClasTool Makefile}

%%%%%%%%%%%%%%%%%%%%%%%%%%%%%%%%%%%%%%%%%%%%%%%%%%%%%%%%%%%%%%%%%%%%%%%%%%%

\begin{document}
\begin{frame}
  \begin{center}
    \begin{tabular*}{\textwidth}%
      {>{\centering}m{0.1\textwidth}@{\extracolsep{\fill}}>{\centering}m{3in}%
      >{\centering}m{0.1\textwidth}}
      \includegraphics[height=0.09\textwidth]{usm} &
      \textsc{\scriptsize{Universidad T\'ecnica Federico Santa Mar\'ia\\
      Departamento de F\'isica}\\
      \footnotesize{UTFSM ATLAS Grid Team}} &
      \includegraphics[height=0.1\textwidth]{root}
    \end{tabular*}

    \vspace{1.4cm}
    \large{\textbf{Nonrecursive Makefile for ClasTool}}

    \vspace{0.8cm}
    \normalsize{Sebasti\'an Mancilla Matta}\\
    \url{smancill@alumnos.inf.utfsm.cl}
  \end{center}
\end{frame}

%%%%%%%%%%%%%%%%%%%%%%%%%%%%%%%%%%%%%%%%%%%%%%%%%%%%%%%%%%%%%%%%%%%%%%%%%%%

\begin{frame}
  \frametitle{Recursive Makefile issues}
  Recursive makefiles are bad primarily because they partition the dependency
  tree into several trees.

  \begin{itemize}
    \item This prevents dependencies between make instances from being
      expressed correctly.\\[2mm]
    \item Also causes (parts of) the dependency tree to be recalculated
      multiple times, which is a performance issue in the end.
  \end{itemize}

  \vspace{5mm}
  This affects both phases of the operation of make:

  \begin{itemize}
    \item It causes make to construct an inaccurate DAG.\\[2mm]
    \item It forces make to traverse the DAG in an inappropriate order.
  \end{itemize}
\end{frame}

%%%%%%%%%%%%%%%%%%%%%%%%%%%%%%%%%%%%%%%%%%%%%%%%%%%%%%%%%%%%%%%%%%%%%%%%%%%

\begin{frame}
  \frametitle{Recursive Makefile issues}
  Other problems include:

  \begin{itemize}
    \item Recursive makes which take \emph{forever} to work out that they need
      to do nothing.\\[2mm]
    \item Recursive makes which do too much, or too little.\\[2mm]
    \item Recursive makes which are overly sensitive to changes in the source
      code and require constant Makefile intervention to keep them
      working.\\[2mm]
    \item It is very hard to get the order of the recursion into the
      sub-directories correct.\\[2mm]
    \item It is often necessary to do more than one pass over the
      sub-directories to build the whole system.
  \end{itemize}
\end{frame}

%%%%%%%%%%%%%%%%%%%%%%%%%%%%%%%%%%%%%%%%%%%%%%%%%%%%%%%%%%%%%%%%%%%%%%%%%%%

\begin{frame}
  \frametitle{Nonrecursive Makefiles}
  \begin{itemize}
    \item The use of a whole project make is not as difficult to put into
      practice as it may at first appear.\\[5mm]
    \item It solves the problems found in the recursive make approach.

      \begin{itemize}
      \item It handles dependencies correctly. Treating the whole tree as a
        single entity is really the right way.\\[1mm]
      \item Specific targets can be built (say a particular program)
        correctly, since make always creates a full dependency graph.\\[1mm]
      \item It is faster.\\[5mm]
      \end{itemize}
    \item It requires a paradigm shift for developers used to recursive
      make.
  \end{itemize}
\end{frame}

%%%%%%%%%%%%%%%%%%%%%%%%%%%%%%%%%%%%%%%%%%%%%%%%%%%%%%%%%%%%%%%%%%%%%%%%%%%%%%

\begin{frame}
  \frametitle{ROOT}
  \begin{itemize}
    \item The ROOT Makefile itself was structured as described in the paper
      \emph{Recursive Make Considered Harmful}.\\[5mm]
    \item The main philosophy is that it is better to have a single large
      Makefile describing the entire project than many small Makefiles, one
      for each sub-project, that are recursively called from the main
      Makefile. By cleverly using the include mechanism the single Makefile
      solution is as modular as the recursive approach without the problems of
      incomplete dependency graphs.\\[5mm]
    \item The single Makefile is fast
      \begin{itemize}
        \item About 1 second to check if anything needs to be recompiled.
      \end{itemize}
  \end{itemize}
\end{frame}

%%%%%%%%%%%%%%%%%%%%%%%%%%%%%%%%%%%%%%%%%%%%%%%%%%%%%%%%%%%%%%%%%%%%%%%%%%%%%%

\begin{frame}
  \frametitle{My approach}
  Three main files in the project root directory:

  \begin{itemize}
    \item \texttt{Makefile\_system:} it has all the system-dependent variables
      definitions used.\\[2mm]
    \item \texttt{Makefile:} the project's nonrecursive Makefile.\\[2mm]
    \item \texttt{Makefile\_templates:} it has all the templates to build the
      specific rules for every directory, and the macros used by the
      \texttt{module.mk} files. 
  \end{itemize}

  \vspace{3mm}
  A \texttt{module.mk} file in every directory for building the libraries.

  \begin{itemize}
    \item All this files are included in the main Makefile.\\[2mm]
    \item All share the same structure.
  \end{itemize}
\end{frame}

%%%%%%%%%%%%%%%%%%%%%%%%%%%%%%%%%%%%%%%%%%%%%%%%%%%%%%%%%%%%%%%%%%%%%%%%%%%

\begin{frame}[t]
  \frametitle{References}

  \begin{thebibliography}{99}
    \bibitem {RMH} \emph{Recursive Make Considered Harmful}, Peter Miller,
      1997\\
      \url{http://miller.emu.id.au/pmiller/books/rmch/}\\[7mm]
    \bibitem {MPM} \emph{Managing Projects with GNU make}, 3rd Edition, Robert
      Mecklenburg, 2004\\
      \url{http://www.makelinux.net/make3/main.html}\\[7mm]
    \bibitem {GIT} My Git repository (here you can find the sources)\\
      \url{http://github.com/smancill/ClasTool_Makefile}
  \end{thebibliography}
\end{frame}

%%%%%%%%%%%%%%%%%%%%%%%%%%%%%%%%%%%%%%%%%%%%%%%%%%%%%%%%%%%%%%%%%%%%%%%%%%%%%%

\end{document}
