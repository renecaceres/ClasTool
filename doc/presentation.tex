\documentclass[11pt]{beamer}
\usepackage[spanish]{babel}
\usepackage[utf8]{inputenc}
\usepackage{graphics}                                
\usepackage{listings}
\usepackage{array}
\usepackage{times}


\usetheme{CambridgeUS}
\usecolortheme{seagull}
\usefonttheme[onlylarge]{structurebold}

\lstset{%
     showstringspaces = false,
     basicstyle=\small\ttfamily,
   }

\lstdefinestyle{Pascal}
   {language=Pascal,
   }

\author{Sebasti\'an Mancilla}
\date{\today}
\title{ClasTool Makefile}

%%%%%%%%%%%%%%%%%%%%%%%%%%%%%%%%%%%%%%%%%%%%%%%%%%%%%%%%%%%%%%%%%%%%%%%%%%%  

\begin{document}
\begin{frame}
	\begin{center}
		\begin{tabular*}{\textwidth}%
      {>{\centering}m{0.1\textwidth}@{\extracolsep{\fill}}>{\centering}m{3in}%
      >{\centering}m{0.1\textwidth}}
			\includegraphics[height=0.09\textwidth]{usm} &
		  \textsc{\scriptsize{Universidad T\'ecnica Federico Santa Mar\'ia\\
      Departamento de F\'isica}\\
      \footnotesize{UTFSM ATLAS Grid Team}} &
			\includegraphics[height=0.1\textwidth]{root} 
		\end{tabular*}

    \vspace{1.4cm}
    \large{\textbf{Nonrecursive Makefile for ClasTool}}

    \vspace{0.8cm}
    \normalsize{Sebasti\'an Mancilla Matta}\\
    \url{smancill@alumnos.inf.utfsm.cl}
	\end{center}
\end{frame}

%%%%%%%%%%%%%%%%%%%%%%%%%%%%%%%%%%%%%%%%%%%%%%%%%%%%%%%%%%%%%%%%%%%%%%%%%%%

\begin{frame}
  \frametitle{Recursive Makefile issues}
  Recursive makefiles are bad primarily because they partition the dependency
  tree into several trees.
  
  \begin{itemize}
    \item This prevents dependencies between make instances from being
      expressed correctly.
    \item Also causes (parts of) the dependency tree to be recalculated
      multiple times, which is a performance issue in the end.
  \end{itemize}
\end{frame}

%%%%%%%%%%%%%%%%%%%%%%%%%%%%%%%%%%%%%%%%%%%%%%%%%%%%%%%%%%%%%%%%%%%%%%%%%%%

\begin{frame}
  \frametitle{ROOT}
  \begin{itemize}
    \item The ROOT Makefile itself was structured as described in the paper:
      ``Recursive Make Considered Harmful''.\\[5mm]
    \item The main philosophy is that it is better to have a single large
      Makefile describing the entire project than many small Makefiles, one
      for each sub-project, that are recursively called from the main
      Makefile. By cleverly using the include mechanism the single Makefile
      solution is as modular as the recursive approach without the problems of
      incomplete dependency graphs.\\[5mm]
    \item The single Makefile is fast
      \begin{itemize}
        \item About 1 second to check if anything needs to be recompiled.
      \end{itemize}
  \end{itemize}
\end{frame}

%%%%%%%%%%%%%%%%%%%%%%%%%%%%%%%%%%%%%%%%%%%%%%%%%%%%%%%%%%%%%%%%%%%%%%%%%%%%%%

\begin{frame}
  \frametitle{My approach}
  Three main files in the project root directory:

  \begin{itemize}
    \item \texttt{Makefile\_system:} it has all the system-dependent variables
      definitions used.\\[2mm]
    \item \texttt{Makefile:} the project's non-recursive Makefile.\\[2mm]
    \item \texttt{Makefile\_templates:} it has all the templates to build the
      specific rules for every directory, and the macros used by
      \texttt{module.mk} files. 
  \end{itemize}
  
  \vspace{3mm}
  A \texttt{module.mk} file in every directory for building the libraries.

  \begin{itemize}
    \item All this files are included in the main Makefile.\\[2mm]
    \item All share the same structure.
  \end{itemize}
\end{frame}

%%%%%%%%%%%%%%%%%%%%%%%%%%%%%%%%%%%%%%%%%%%%%%%%%%%%%%%%%%%%%%%%%%%%%%%%%%%%%%

\end{document}
